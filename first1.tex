% Options for packages loaded elsewhere
\PassOptionsToPackage{unicode}{hyperref}
\PassOptionsToPackage{hyphens}{url}
%
\documentclass[
]{article}
\usepackage{amsmath,amssymb}
\usepackage{iftex}
\ifPDFTeX
  \usepackage[T1]{fontenc}
  \usepackage[utf8]{inputenc}
  \usepackage{textcomp} % provide euro and other symbols
\else % if luatex or xetex
  \usepackage{unicode-math} % this also loads fontspec
  \defaultfontfeatures{Scale=MatchLowercase}
  \defaultfontfeatures[\rmfamily]{Ligatures=TeX,Scale=1}
\fi
\usepackage{lmodern}
\ifPDFTeX\else
  % xetex/luatex font selection
\fi
% Use upquote if available, for straight quotes in verbatim environments
\IfFileExists{upquote.sty}{\usepackage{upquote}}{}
\IfFileExists{microtype.sty}{% use microtype if available
  \usepackage[]{microtype}
  \UseMicrotypeSet[protrusion]{basicmath} % disable protrusion for tt fonts
}{}
\makeatletter
\@ifundefined{KOMAClassName}{% if non-KOMA class
  \IfFileExists{parskip.sty}{%
    \usepackage{parskip}
  }{% else
    \setlength{\parindent}{0pt}
    \setlength{\parskip}{6pt plus 2pt minus 1pt}}
}{% if KOMA class
  \KOMAoptions{parskip=half}}
\makeatother
\usepackage{xcolor}
\usepackage[margin=1in]{geometry}
\usepackage{color}
\usepackage{fancyvrb}
\newcommand{\VerbBar}{|}
\newcommand{\VERB}{\Verb[commandchars=\\\{\}]}
\DefineVerbatimEnvironment{Highlighting}{Verbatim}{commandchars=\\\{\}}
% Add ',fontsize=\small' for more characters per line
\usepackage{framed}
\definecolor{shadecolor}{RGB}{248,248,248}
\newenvironment{Shaded}{\begin{snugshade}}{\end{snugshade}}
\newcommand{\AlertTok}[1]{\textcolor[rgb]{0.94,0.16,0.16}{#1}}
\newcommand{\AnnotationTok}[1]{\textcolor[rgb]{0.56,0.35,0.01}{\textbf{\textit{#1}}}}
\newcommand{\AttributeTok}[1]{\textcolor[rgb]{0.13,0.29,0.53}{#1}}
\newcommand{\BaseNTok}[1]{\textcolor[rgb]{0.00,0.00,0.81}{#1}}
\newcommand{\BuiltInTok}[1]{#1}
\newcommand{\CharTok}[1]{\textcolor[rgb]{0.31,0.60,0.02}{#1}}
\newcommand{\CommentTok}[1]{\textcolor[rgb]{0.56,0.35,0.01}{\textit{#1}}}
\newcommand{\CommentVarTok}[1]{\textcolor[rgb]{0.56,0.35,0.01}{\textbf{\textit{#1}}}}
\newcommand{\ConstantTok}[1]{\textcolor[rgb]{0.56,0.35,0.01}{#1}}
\newcommand{\ControlFlowTok}[1]{\textcolor[rgb]{0.13,0.29,0.53}{\textbf{#1}}}
\newcommand{\DataTypeTok}[1]{\textcolor[rgb]{0.13,0.29,0.53}{#1}}
\newcommand{\DecValTok}[1]{\textcolor[rgb]{0.00,0.00,0.81}{#1}}
\newcommand{\DocumentationTok}[1]{\textcolor[rgb]{0.56,0.35,0.01}{\textbf{\textit{#1}}}}
\newcommand{\ErrorTok}[1]{\textcolor[rgb]{0.64,0.00,0.00}{\textbf{#1}}}
\newcommand{\ExtensionTok}[1]{#1}
\newcommand{\FloatTok}[1]{\textcolor[rgb]{0.00,0.00,0.81}{#1}}
\newcommand{\FunctionTok}[1]{\textcolor[rgb]{0.13,0.29,0.53}{\textbf{#1}}}
\newcommand{\ImportTok}[1]{#1}
\newcommand{\InformationTok}[1]{\textcolor[rgb]{0.56,0.35,0.01}{\textbf{\textit{#1}}}}
\newcommand{\KeywordTok}[1]{\textcolor[rgb]{0.13,0.29,0.53}{\textbf{#1}}}
\newcommand{\NormalTok}[1]{#1}
\newcommand{\OperatorTok}[1]{\textcolor[rgb]{0.81,0.36,0.00}{\textbf{#1}}}
\newcommand{\OtherTok}[1]{\textcolor[rgb]{0.56,0.35,0.01}{#1}}
\newcommand{\PreprocessorTok}[1]{\textcolor[rgb]{0.56,0.35,0.01}{\textit{#1}}}
\newcommand{\RegionMarkerTok}[1]{#1}
\newcommand{\SpecialCharTok}[1]{\textcolor[rgb]{0.81,0.36,0.00}{\textbf{#1}}}
\newcommand{\SpecialStringTok}[1]{\textcolor[rgb]{0.31,0.60,0.02}{#1}}
\newcommand{\StringTok}[1]{\textcolor[rgb]{0.31,0.60,0.02}{#1}}
\newcommand{\VariableTok}[1]{\textcolor[rgb]{0.00,0.00,0.00}{#1}}
\newcommand{\VerbatimStringTok}[1]{\textcolor[rgb]{0.31,0.60,0.02}{#1}}
\newcommand{\WarningTok}[1]{\textcolor[rgb]{0.56,0.35,0.01}{\textbf{\textit{#1}}}}
\usepackage{graphicx}
\makeatletter
\def\maxwidth{\ifdim\Gin@nat@width>\linewidth\linewidth\else\Gin@nat@width\fi}
\def\maxheight{\ifdim\Gin@nat@height>\textheight\textheight\else\Gin@nat@height\fi}
\makeatother
% Scale images if necessary, so that they will not overflow the page
% margins by default, and it is still possible to overwrite the defaults
% using explicit options in \includegraphics[width, height, ...]{}
\setkeys{Gin}{width=\maxwidth,height=\maxheight,keepaspectratio}
% Set default figure placement to htbp
\makeatletter
\def\fps@figure{htbp}
\makeatother
\setlength{\emergencystretch}{3em} % prevent overfull lines
\providecommand{\tightlist}{%
  \setlength{\itemsep}{0pt}\setlength{\parskip}{0pt}}
\setcounter{secnumdepth}{-\maxdimen} % remove section numbering
\usepackage{booktabs}
\usepackage{longtable}
\usepackage{array}
\usepackage{multirow}
\usepackage{wrapfig}
\usepackage{float}
\usepackage{colortbl}
\usepackage{pdflscape}
\usepackage{tabu}
\usepackage{threeparttable}
\usepackage{threeparttablex}
\usepackage[normalem]{ulem}
\usepackage{makecell}
\usepackage{xcolor}
\ifLuaTeX
  \usepackage{selnolig}  % disable illegal ligatures
\fi
\usepackage{bookmark}
\IfFileExists{xurl.sty}{\usepackage{xurl}}{} % add URL line breaks if available
\urlstyle{same}
\hypersetup{
  pdftitle={try1},
  pdfauthor={Sourav Jose},
  hidelinks,
  pdfcreator={LaTeX via pandoc}}

\title{try1}
\author{Sourav Jose}
\date{2025-03-26}

\begin{document}
\maketitle

\subsection{R Markdown}\label{r-markdown}

This is an R Markdown document. Markdown is a simple formatting syntax
for authoring HTML, PDF, and MS Word documents. For more details on
using R Markdown see \url{http://rmarkdown.rstudio.com}.

When you click the \textbf{Knit} button a document will be generated
that includes both content as well as the output of any embedded R code
chunks within the document. You can embed an R code chunk like this:

\subsection{Including Plots}\label{including-plots}

\begin{Shaded}
\begin{Highlighting}[]
\CommentTok{\#loading the data and libraries }
\FunctionTok{library}\NormalTok{(learningtower)}
\end{Highlighting}
\end{Shaded}

\begin{verbatim}
## The learningtower package (version 1.1.0) provides data from OECD PISA database between 2000 - 2022. For package size reasons, only a small subset is provided in the package. Use the function `load_student()` to access the full data.
\end{verbatim}

\begin{Shaded}
\begin{Highlighting}[]
\FunctionTok{library}\NormalTok{(patchwork)}
\CommentTok{\# Load libraries}
\FunctionTok{library}\NormalTok{(tidyverse)}
\end{Highlighting}
\end{Shaded}

\begin{verbatim}
## -- Attaching core tidyverse packages ------------------------ tidyverse 2.0.0 --
## v dplyr     1.1.4     v readr     2.1.5
## v forcats   1.0.0     v stringr   1.5.1
## v ggplot2   3.5.1     v tibble    3.2.1
## v lubridate 1.9.4     v tidyr     1.3.1
## v purrr     1.0.4
\end{verbatim}

\begin{verbatim}
## -- Conflicts ------------------------------------------ tidyverse_conflicts() --
## x dplyr::filter() masks stats::filter()
## x dplyr::lag()    masks stats::lag()
## i Use the conflicted package (<http://conflicted.r-lib.org/>) to force all conflicts to become errors
\end{verbatim}

\begin{Shaded}
\begin{Highlighting}[]
\FunctionTok{library}\NormalTok{(ggplot2)}
\FunctionTok{library}\NormalTok{(cluster)       }\CommentTok{\# For clustering}
    \CommentTok{\# For visualizing clusters}
       \CommentTok{\# For machine learning}
\FunctionTok{library}\NormalTok{(randomForest)   }\CommentTok{\# For Random Forest}
\end{Highlighting}
\end{Shaded}

\begin{verbatim}
## randomForest 4.7-1.2
## Type rfNews() to see new features/changes/bug fixes.
## 
## Attaching package: 'randomForest'
## 
## The following object is masked from 'package:dplyr':
## 
##     combine
## 
## The following object is masked from 'package:ggplot2':
## 
##     margin
\end{verbatim}

\begin{Shaded}
\begin{Highlighting}[]
\FunctionTok{library}\NormalTok{(e1071)          }\CommentTok{\# For SVM}
      \CommentTok{\# For XGBoost}
\FunctionTok{library}\NormalTok{(plotly)         }\CommentTok{\# For interactive plots}
\end{Highlighting}
\end{Shaded}

\begin{verbatim}
## 
## Attaching package: 'plotly'
## 
## The following object is masked from 'package:ggplot2':
## 
##     last_plot
## 
## The following object is masked from 'package:stats':
## 
##     filter
## 
## The following object is masked from 'package:graphics':
## 
##     layout
\end{verbatim}

\begin{Shaded}
\begin{Highlighting}[]
\FunctionTok{library}\NormalTok{(xgboost)  }
\end{Highlighting}
\end{Shaded}

\begin{verbatim}
## 
## Attaching package: 'xgboost'
## 
## The following object is masked from 'package:plotly':
## 
##     slice
## 
## The following object is masked from 'package:dplyr':
## 
##     slice
\end{verbatim}

\begin{Shaded}
\begin{Highlighting}[]
\FunctionTok{library}\NormalTok{(caret)   }
\end{Highlighting}
\end{Shaded}

\begin{verbatim}
## Loading required package: lattice
## 
## Attaching package: 'caret'
## 
## The following object is masked from 'package:purrr':
## 
##     lift
\end{verbatim}

\begin{Shaded}
\begin{Highlighting}[]
\FunctionTok{library}\NormalTok{(factoextra) }
\end{Highlighting}
\end{Shaded}

\begin{verbatim}
## Welcome! Want to learn more? See two factoextra-related books at https://goo.gl/ve3WBa
\end{verbatim}

\begin{Shaded}
\begin{Highlighting}[]
\FunctionTok{library}\NormalTok{(gghighlight)}

\FunctionTok{library}\NormalTok{(ggrepel)}

\FunctionTok{library}\NormalTok{(tsibble)}
\end{Highlighting}
\end{Shaded}

\begin{verbatim}
## Registered S3 method overwritten by 'tsibble':
##   method               from 
##   as_tibble.grouped_df dplyr
## 
## Attaching package: 'tsibble'
## 
## The following object is masked from 'package:lubridate':
## 
##     interval
## 
## The following objects are masked from 'package:base':
## 
##     intersect, setdiff, union
\end{verbatim}

\begin{Shaded}
\begin{Highlighting}[]
\FunctionTok{library}\NormalTok{(kableExtra)}
\end{Highlighting}
\end{Shaded}

\begin{verbatim}
## 
## Attaching package: 'kableExtra'
## 
## The following object is masked from 'package:dplyr':
## 
##     group_rows
\end{verbatim}

\begin{Shaded}
\begin{Highlighting}[]
\FunctionTok{library}\NormalTok{(broom)}
\NormalTok{student }\OtherTok{\textless{}{-}} \FunctionTok{load\_student}\NormalTok{(}\StringTok{"all"}\NormalTok{)}
\end{Highlighting}
\end{Shaded}

\begin{verbatim}
## Downloading year 2000...
## 
## Downloading year 2003...
## 
## Downloading year 2006...
## 
## Downloading year 2009...
## 
## Downloading year 2012...
## 
## Downloading year 2015...
## 
## Downloading year 2018...
## 
## Downloading year 2022...
\end{verbatim}

\begin{Shaded}
\begin{Highlighting}[]
\FunctionTok{data}\NormalTok{(countrycode)}

\FunctionTok{theme\_set}\NormalTok{(}\FunctionTok{theme\_classic}\NormalTok{(}\DecValTok{18}\NormalTok{) }\SpecialCharTok{+}
            \FunctionTok{theme}\NormalTok{(}\AttributeTok{legend.position =} \StringTok{"bottom"}\NormalTok{))}
\end{Highlighting}
\end{Shaded}

\begin{Shaded}
\begin{Highlighting}[]
\NormalTok{ireland\_data }\OtherTok{\textless{}{-}}\NormalTok{ student }\SpecialCharTok{\%\textgreater{}\%} \FunctionTok{filter}\NormalTok{(country }\SpecialCharTok{==} \StringTok{"IRL"}\NormalTok{)}
\NormalTok{ireland\_data}
\end{Highlighting}
\end{Shaded}

\begin{verbatim}
## # A tibble: 36,439 x 22
##     year country school_id student_id mother_educ father_educ gender computer
##    <int> <fct>   <chr>          <int> <fct>       <fct>       <fct>  <fct>   
##  1  2000 IRL     1014               2 <NA>        <NA>        female <NA>    
##  2  2000 IRL     1014               3 <NA>        <NA>        female <NA>    
##  3  2000 IRL     1014               4 <NA>        <NA>        female <NA>    
##  4  2000 IRL     1014               5 <NA>        <NA>        female <NA>    
##  5  2000 IRL     1014               6 <NA>        <NA>        female <NA>    
##  6  2000 IRL     1014               7 <NA>        <NA>        female <NA>    
##  7  2000 IRL     1014              10 <NA>        <NA>        female <NA>    
##  8  2000 IRL     1014              12 <NA>        <NA>        female <NA>    
##  9  2000 IRL     1014              13 <NA>        <NA>        female <NA>    
## 10  2000 IRL     1014              14 <NA>        <NA>        female <NA>    
## # i 36,429 more rows
## # i 14 more variables: internet <fct>, math <dbl>, read <dbl>, science <dbl>,
## #   stu_wgt <dbl>, desk <fct>, room <fct>, dishwasher <fct>, television <fct>,
## #   computer_n <fct>, car <fct>, book <fct>, wealth <dbl>, escs <dbl>
\end{verbatim}

\begin{Shaded}
\begin{Highlighting}[]
\NormalTok{years\_in\_data }\OtherTok{\textless{}{-}} \FunctionTok{unique}\NormalTok{(ireland\_data}\SpecialCharTok{$}\NormalTok{year)}
\FunctionTok{print}\NormalTok{(years\_in\_data)}
\end{Highlighting}
\end{Shaded}

\begin{verbatim}
## [1] 2000 2003 2006 2009 2012 2015 2018 2022
\end{verbatim}

\begin{Shaded}
\begin{Highlighting}[]
\CommentTok{\# Scatter plot with regression line}
\FunctionTok{plot}\NormalTok{(ireland\_data}\SpecialCharTok{$}\NormalTok{escs, ireland\_data}\SpecialCharTok{$}\NormalTok{math,}
     \AttributeTok{main =} \StringTok{"Scatter Plot of ESCS and Math Scores with Regression Line"}\NormalTok{,}
     \AttributeTok{xlab =} \StringTok{"ESCS"}\NormalTok{, }\AttributeTok{ylab =} \StringTok{"Math Score"}\NormalTok{,}
     \AttributeTok{pch =} \DecValTok{19}\NormalTok{, }\AttributeTok{col =} \FunctionTok{rgb}\NormalTok{(}\DecValTok{0}\NormalTok{, }\DecValTok{0}\NormalTok{, }\DecValTok{1}\NormalTok{, }\FloatTok{0.5}\NormalTok{))}
\FunctionTok{abline}\NormalTok{(}\FunctionTok{lm}\NormalTok{(math }\SpecialCharTok{\textasciitilde{}}\NormalTok{ escs, }\AttributeTok{data =}\NormalTok{ ireland\_data), }\AttributeTok{col =} \StringTok{"red"}\NormalTok{, }\AttributeTok{lwd =} \DecValTok{2}\NormalTok{)  }\CommentTok{\# Add red regression line}
\end{Highlighting}
\end{Shaded}

\includegraphics{first1_files/figure-latex/unnamed-chunk-6-1.pdf} The
plot shows that socioeconomic status (as measured by ESCS) is an
important factor influencing math performance. Students from more
affluent backgrounds tend to score higher on math assessments

\begin{Shaded}
\begin{Highlighting}[]
\NormalTok{ireland\_data }\SpecialCharTok{\%\textgreater{}\%}
  \FunctionTok{group\_by}\NormalTok{(year, gender) }\SpecialCharTok{\%\textgreater{}\%}
  \FunctionTok{summarise}\NormalTok{(}\AttributeTok{mean\_math =} \FunctionTok{mean}\NormalTok{(math, }\AttributeTok{na.rm =} \ConstantTok{TRUE}\NormalTok{)) }\SpecialCharTok{\%\textgreater{}\%}
  \FunctionTok{ggplot}\NormalTok{(}\FunctionTok{aes}\NormalTok{(}\AttributeTok{x =}\NormalTok{ year, }\AttributeTok{y =}\NormalTok{ mean\_math, }\AttributeTok{color =}\NormalTok{ gender)) }\SpecialCharTok{+}
  \FunctionTok{geom\_line}\NormalTok{() }\SpecialCharTok{+} \FunctionTok{geom\_point}\NormalTok{() }\SpecialCharTok{+}
  \FunctionTok{labs}\NormalTok{(}\AttributeTok{title =} \StringTok{"Math Score Trends by Gender"}\NormalTok{, }\AttributeTok{y =} \StringTok{"Mean Math Score"}\NormalTok{, }\AttributeTok{x =} \StringTok{"Year"}\NormalTok{)}
\end{Highlighting}
\end{Shaded}

\begin{verbatim}
## `summarise()` has grouped output by 'year'. You can override using the
## `.groups` argument.
\end{verbatim}

\includegraphics{first1_files/figure-latex/unnamed-chunk-7-1.pdf}

The graph clearly shows that, in most years, male students consistently
outperform female students in PISA math assessments. This difference in
average scores is observable at every time point, with male students
typically showing higher average scores than their female counterparts.

\begin{Shaded}
\begin{Highlighting}[]
\CommentTok{\# Calculate the average math score by number of books}
\NormalTok{avg\_math\_books }\OtherTok{\textless{}{-}} \FunctionTok{aggregate}\NormalTok{(math }\SpecialCharTok{\textasciitilde{}}\NormalTok{ book, }\AttributeTok{data =}\NormalTok{ ireland\_data, }\AttributeTok{FUN =}\NormalTok{ mean)}

\CommentTok{\# Bar plot for average math scores by number of books}
\FunctionTok{barplot}\NormalTok{(avg\_math\_books}\SpecialCharTok{$}\NormalTok{math, }\AttributeTok{names.arg =}\NormalTok{ avg\_math\_books}\SpecialCharTok{$}\NormalTok{book,}
        \AttributeTok{col =} \StringTok{"lightblue"}\NormalTok{, }\AttributeTok{main =} \StringTok{"Average Math Scores by Number of Books"}\NormalTok{,}
        \AttributeTok{xlab =} \StringTok{"Number of Books"}\NormalTok{, }\AttributeTok{ylab =} \StringTok{"Average Math Score"}\NormalTok{)}
\end{Highlighting}
\end{Shaded}

\includegraphics{first1_files/figure-latex/unnamed-chunk-8-1.pdf}

The plot suggests that having access to books does not drastically
change average math scores, but there might be a small advantage for
students with a moderate number of books. The slight drop for the
``none'' category suggests that having no books could be a disadvantage,
although the difference is subtle.

\begin{Shaded}
\begin{Highlighting}[]
\CommentTok{\# Bar plot for math scores by television access}
\NormalTok{aggregate\_math\_tv }\OtherTok{\textless{}{-}} \FunctionTok{aggregate}\NormalTok{(math }\SpecialCharTok{\textasciitilde{}}\NormalTok{ television, }\AttributeTok{data =}\NormalTok{ ireland\_data, }\AttributeTok{FUN =}\NormalTok{ mean)}
\FunctionTok{barplot}\NormalTok{(aggregate\_math\_tv}\SpecialCharTok{$}\NormalTok{math, }\AttributeTok{names.arg =}\NormalTok{ aggregate\_math\_tv}\SpecialCharTok{$}\NormalTok{television,}
        \AttributeTok{col =} \StringTok{"lightyellow"}\NormalTok{, }\AttributeTok{main =} \StringTok{"Average Math Scores by Television Access"}\NormalTok{,}
        \AttributeTok{xlab =} \StringTok{"Television Access"}\NormalTok{, }\AttributeTok{ylab =} \StringTok{"Average Math Score"}\NormalTok{)}
\end{Highlighting}
\end{Shaded}

\includegraphics{first1_files/figure-latex/unnamed-chunk-9-1.pdf}

All four bars have a similar height, indicating that there is no
significant difference in the average math scores between students who
have varying numbers of televisions in their households.

The data seems to show that the number of televisions does not appear to
have a strong relationship with the average math score in this dataset.

\begin{Shaded}
\begin{Highlighting}[]
\CommentTok{\# Boxplot of math scores by computer access}
\FunctionTok{ggplot}\NormalTok{(ireland\_data, }\FunctionTok{aes}\NormalTok{(}\AttributeTok{x =} \FunctionTok{factor}\NormalTok{(computer\_n), }\AttributeTok{y =}\NormalTok{ math, }\AttributeTok{fill =} \FunctionTok{factor}\NormalTok{(computer\_n))) }\SpecialCharTok{+}
  \FunctionTok{geom\_boxplot}\NormalTok{() }\SpecialCharTok{+}
  \FunctionTok{scale\_fill\_manual}\NormalTok{(}\AttributeTok{values =} \FunctionTok{c}\NormalTok{(}\StringTok{\textquotesingle{}blue\textquotesingle{}}\NormalTok{, }\StringTok{\textquotesingle{}red\textquotesingle{}}\NormalTok{, }\StringTok{\textquotesingle{}green\textquotesingle{}}\NormalTok{, }\StringTok{\textquotesingle{}yellow\textquotesingle{}}\NormalTok{))}\SpecialCharTok{+}
  \FunctionTok{theme\_minimal}\NormalTok{() }\SpecialCharTok{+}
  \FunctionTok{labs}\NormalTok{(}\AttributeTok{title =} \StringTok{"Math Scores Distribution by Computer Access"}\NormalTok{,}
       \AttributeTok{x =} \StringTok{"Computer Access (0 = No, 1 = Yes)"}\NormalTok{,}
       \AttributeTok{y =} \StringTok{"Math Scores"}\NormalTok{) }\SpecialCharTok{+}
  \FunctionTok{scale\_x\_discrete}\NormalTok{(}\AttributeTok{labels =} \FunctionTok{c}\NormalTok{(}\StringTok{"Without Computer"}\NormalTok{, }\StringTok{"With Computer"}\NormalTok{))}
\end{Highlighting}
\end{Shaded}

\begin{verbatim}
## Warning: Removed 1285 rows containing non-finite outside the scale range
## (`stat_boxplot()`).
\end{verbatim}

\includegraphics{first1_files/figure-latex/unnamed-chunk-10-1.pdf} The
histogram suggests that students with computer access tend to have a
more evenly distributed performance across various math score ranges,
with some achieving high scores.

On the other hand, students without computer access are more
concentrated in the lower to mid-range math scores, indicating that lack
of computer access might be associated with lower math performance.

\begin{Shaded}
\begin{Highlighting}[]
\NormalTok{ireland\_data }\SpecialCharTok{\%\textgreater{}\%}
  \FunctionTok{group\_by}\NormalTok{(year, internet) }\SpecialCharTok{\%\textgreater{}\%}
  \FunctionTok{summarise}\NormalTok{(}\AttributeTok{mean\_math =} \FunctionTok{mean}\NormalTok{(math, }\AttributeTok{na.rm =} \ConstantTok{TRUE}\NormalTok{)) }\SpecialCharTok{\%\textgreater{}\%}
  \FunctionTok{ggplot}\NormalTok{(}\FunctionTok{aes}\NormalTok{(}\AttributeTok{x =}\NormalTok{ year, }\AttributeTok{y =}\NormalTok{ mean\_math, }\AttributeTok{color =}\NormalTok{ internet)) }\SpecialCharTok{+}
  \FunctionTok{geom\_line}\NormalTok{() }\SpecialCharTok{+} \FunctionTok{geom\_point}\NormalTok{() }\SpecialCharTok{+}
  \FunctionTok{labs}\NormalTok{(}\AttributeTok{title =} \StringTok{"Math Scores by Internet Access Over Years"}\NormalTok{, }\AttributeTok{y =} \StringTok{"Mean Math Score"}\NormalTok{, }\AttributeTok{x =} \StringTok{"Year"}\NormalTok{)}
\end{Highlighting}
\end{Shaded}

\begin{verbatim}
## `summarise()` has grouped output by 'year'. You can override using the
## `.groups` argument.
\end{verbatim}

\includegraphics{first1_files/figure-latex/unnamed-chunk-12-1.pdf}

\begin{Shaded}
\begin{Highlighting}[]
\CommentTok{\# Count of students with and without internet access}
\NormalTok{internet\_count }\OtherTok{\textless{}{-}} \FunctionTok{table}\NormalTok{(ireland\_data}\SpecialCharTok{$}\NormalTok{internet)}

\CommentTok{\# Pie chart of internet access distribution}
\FunctionTok{pie}\NormalTok{(internet\_count, }\AttributeTok{labels =} \FunctionTok{names}\NormalTok{(internet\_count), }\AttributeTok{col =} \FunctionTok{c}\NormalTok{(}\StringTok{"lightgreen"}\NormalTok{, }\StringTok{"lightblue"}\NormalTok{), }
    \AttributeTok{main =} \StringTok{"Distribution of Internet Access"}\NormalTok{)}
\end{Highlighting}
\end{Shaded}

\includegraphics{first1_files/figure-latex/unnamed-chunk-12-2.pdf} The
plot shows that internet access does not seem to drastically improve
math scores over time. However, students with internet access tend to
score slightly higher than those without internet access.

\begin{Shaded}
\begin{Highlighting}[]
\CommentTok{\# Calculate average math scores by gender}
\NormalTok{avg\_math\_by\_gender }\OtherTok{\textless{}{-}} \FunctionTok{aggregate}\NormalTok{(math }\SpecialCharTok{\textasciitilde{}}\NormalTok{ gender, }\AttributeTok{data =}\NormalTok{ ireland\_data, }\AttributeTok{FUN =}\NormalTok{ mean)}

\CommentTok{\# Bar plot of average math scores by gender}
\FunctionTok{barplot}\NormalTok{(avg\_math\_by\_gender}\SpecialCharTok{$}\NormalTok{math, }\AttributeTok{names.arg =}\NormalTok{ avg\_math\_by\_gender}\SpecialCharTok{$}\NormalTok{gender, }
        \AttributeTok{col =} \FunctionTok{c}\NormalTok{(}\StringTok{"pink"}\NormalTok{, }\StringTok{"lightblue"}\NormalTok{), }\AttributeTok{main =} \StringTok{"Average Math Scores by Gender"}\NormalTok{, }
        \AttributeTok{xlab =} \StringTok{"Gender"}\NormalTok{, }\AttributeTok{ylab =} \StringTok{"Average Math Score"}\NormalTok{)}
\end{Highlighting}
\end{Shaded}

\includegraphics{first1_files/figure-latex/unnamed-chunk-15-1.pdf}

\begin{Shaded}
\begin{Highlighting}[]
\CommentTok{\# Boxplot for father\textquotesingle{}s education vs. math scores}
\FunctionTok{boxplot}\NormalTok{(math }\SpecialCharTok{\textasciitilde{}}\NormalTok{ father\_educ, }\AttributeTok{data =}\NormalTok{ ireland\_data, }
        \AttributeTok{main =} \StringTok{"Math Scores by Father\textquotesingle{}s Education Level"}\NormalTok{,}
        \AttributeTok{xlab =} \StringTok{"Father\textquotesingle{}s Education Level"}\NormalTok{, }\AttributeTok{ylab =} \StringTok{"Math Score"}\NormalTok{,}
        \AttributeTok{col =} \StringTok{"lightcoral"}\NormalTok{, }\AttributeTok{border =} \StringTok{"darkred"}\NormalTok{)}
\end{Highlighting}
\end{Shaded}

\includegraphics{first1_files/figure-latex/unnamed-chunk-16-1.pdf} The
plot suggests that students whose fathers have higher education levels
tend to have higher median math scores. The difference in medians
between ISCED 1 and ISCED 3A, for example, is noticeable.

The presence of outliers in all groups suggests that while the general
trend is that higher paternal education is associated with better math
performance, there are exceptions.

\begin{Shaded}
\begin{Highlighting}[]
\CommentTok{\# Boxplot for mother\textquotesingle{}s education vs. math scores}
\FunctionTok{boxplot}\NormalTok{(math }\SpecialCharTok{\textasciitilde{}}\NormalTok{ mother\_educ, }\AttributeTok{data =}\NormalTok{ ireland\_data, }
        \AttributeTok{main =} \StringTok{"Math Scores by Mother\textquotesingle{}s Education Level"}\NormalTok{,}
        \AttributeTok{xlab =} \StringTok{"Mother\textquotesingle{}s Education Level"}\NormalTok{, }\AttributeTok{ylab =} \StringTok{"Math Score"}\NormalTok{,}
        \AttributeTok{col =} \StringTok{"lightblue"}\NormalTok{, }\AttributeTok{border =} \StringTok{"darkblue"}\NormalTok{)}
\end{Highlighting}
\end{Shaded}

\includegraphics{first1_files/figure-latex/unnamed-chunk-17-1.pdf}

The plot suggests that students whose mothers have higher education
levels tend to have higher math scores. The median score for students
with mothers who have ISCED 3A or ISCED 3B, C is significantly higher
than for those with mothers who have ISCED 1 or ISCED 2.

The presence of outliers in all groups suggests that while the general
trend is that higher maternal education is associated with better math
performance, there are exceptions.

\begin{Shaded}
\begin{Highlighting}[]
\NormalTok{avg\_math\_year }\OtherTok{\textless{}{-}}\NormalTok{ ireland\_data }\SpecialCharTok{\%\textgreater{}\%}
  \FunctionTok{group\_by}\NormalTok{(year) }\SpecialCharTok{\%\textgreater{}\%}
  \FunctionTok{summarise}\NormalTok{(}\AttributeTok{mean\_math =} \FunctionTok{mean}\NormalTok{(math, }\AttributeTok{na.rm =} \ConstantTok{TRUE}\NormalTok{))}

\NormalTok{t}
\end{Highlighting}
\end{Shaded}

\begin{verbatim}
## function (x) 
## UseMethod("t")
## <bytecode: 0x0000027e650be1e8>
## <environment: namespace:base>
\end{verbatim}

\begin{Shaded}
\begin{Highlighting}[]
\FunctionTok{ggplot}\NormalTok{(avg\_math\_year, }\FunctionTok{aes}\NormalTok{(}\AttributeTok{x =}\NormalTok{ year, }\AttributeTok{y =}\NormalTok{ mean\_math)) }\SpecialCharTok{+}
  \FunctionTok{geom\_line}\NormalTok{() }\SpecialCharTok{+}
  \FunctionTok{geom\_point}\NormalTok{() }\SpecialCharTok{+}
  \FunctionTok{theme\_minimal}\NormalTok{() }\SpecialCharTok{+}
  \FunctionTok{labs}\NormalTok{(}\AttributeTok{title =} \StringTok{"Trend of Math Scores Over Years in Ireland"}\NormalTok{, }\AttributeTok{y =} \StringTok{"Average Math Score"}\NormalTok{)}
\end{Highlighting}
\end{Shaded}

\includegraphics{first1_files/figure-latex/unnamed-chunk-18-1.pdf}

\begin{Shaded}
\begin{Highlighting}[]
\FunctionTok{library}\NormalTok{(ggplot2)}

\CommentTok{\# Assuming you have a dataframe \textquotesingle{}ireland\_data\textquotesingle{} with columns \textquotesingle{}read\textquotesingle{} and \textquotesingle{}math\textquotesingle{}}
\FunctionTok{ggplot}\NormalTok{(ireland\_data, }\FunctionTok{aes}\NormalTok{(}\AttributeTok{x =}\NormalTok{ read, }\AttributeTok{y =}\NormalTok{ math)) }\SpecialCharTok{+}
  \FunctionTok{geom\_point}\NormalTok{(}\AttributeTok{color =} \StringTok{\textquotesingle{}blue\textquotesingle{}}\NormalTok{, }\AttributeTok{alpha =} \FloatTok{0.6}\NormalTok{) }\SpecialCharTok{+} \CommentTok{\# Scatter plot}
  \FunctionTok{geom\_smooth}\NormalTok{(}\AttributeTok{method =} \StringTok{\textquotesingle{}lm\textquotesingle{}}\NormalTok{, }\AttributeTok{color =} \StringTok{\textquotesingle{}red\textquotesingle{}}\NormalTok{, }\AttributeTok{se =} \ConstantTok{FALSE}\NormalTok{) }\SpecialCharTok{+} \CommentTok{\# Linear regression line}
  \FunctionTok{labs}\NormalTok{(}\AttributeTok{title =} \StringTok{\textquotesingle{}Relationship Between Reading and Math Scores\textquotesingle{}}\NormalTok{,}
       \AttributeTok{x =} \StringTok{\textquotesingle{}Reading Score\textquotesingle{}}\NormalTok{,}
       \AttributeTok{y =} \StringTok{\textquotesingle{}Math Score\textquotesingle{}}\NormalTok{) }\SpecialCharTok{+}
  \FunctionTok{theme\_minimal}\NormalTok{()}
\end{Highlighting}
\end{Shaded}

\begin{verbatim}
## `geom_smooth()` using formula = 'y ~ x'
\end{verbatim}

\begin{verbatim}
## Warning: Removed 1285 rows containing non-finite outside the scale range
## (`stat_smooth()`).
\end{verbatim}

\begin{verbatim}
## Warning: Removed 1285 rows containing missing values or values outside the scale range
## (`geom_point()`).
\end{verbatim}

\includegraphics{first1_files/figure-latex/unnamed-chunk-20-1.pdf}

\begin{Shaded}
\begin{Highlighting}[]
\CommentTok{\# Remove rows with NA in \textquotesingle{}book\textquotesingle{} column}
\NormalTok{data\_filtered }\OtherTok{\textless{}{-}}\NormalTok{ ireland\_data }\SpecialCharTok{\%\textgreater{}\%} \FunctionTok{filter}\NormalTok{(}\SpecialCharTok{!}\FunctionTok{is.na}\NormalTok{(book))}

\CommentTok{\# Create the boxplot for math score vs. book availability}
\FunctionTok{ggplot}\NormalTok{(data\_filtered, }\FunctionTok{aes}\NormalTok{(}\AttributeTok{x =}\NormalTok{ book, }\AttributeTok{y =}\NormalTok{ math, }\AttributeTok{fill =}\NormalTok{ book)) }\SpecialCharTok{+}
  \FunctionTok{geom\_boxplot}\NormalTok{() }\SpecialCharTok{+}
  \FunctionTok{labs}\NormalTok{(}\AttributeTok{title =} \StringTok{"Distribution of Math Scores by Book Availability"}\NormalTok{,}
       \AttributeTok{x =} \StringTok{"Book Availability"}\NormalTok{,}
       \AttributeTok{y =} \StringTok{"Math Score"}\NormalTok{) }\SpecialCharTok{+}
  \FunctionTok{scale\_fill\_brewer}\NormalTok{(}\AttributeTok{palette =} \StringTok{"Set3"}\NormalTok{) }\SpecialCharTok{+}  \CommentTok{\# Optional: use color palette for distinction}
  \FunctionTok{theme\_minimal}\NormalTok{()}
\end{Highlighting}
\end{Shaded}

\begin{verbatim}
## Warning: Removed 1269 rows containing non-finite outside the scale range
## (`stat_boxplot()`).
\end{verbatim}

\includegraphics{first1_files/figure-latex/unnamed-chunk-21-1.pdf}

\begin{Shaded}
\begin{Highlighting}[]
\FunctionTok{interaction.plot}\NormalTok{(ireland\_data}\SpecialCharTok{$}\NormalTok{internet, ireland\_data}\SpecialCharTok{$}\NormalTok{father\_educ, ireland\_data}\SpecialCharTok{$}\NormalTok{math,}
                 \AttributeTok{xlab =} \StringTok{"Internet Access"}\NormalTok{, }\AttributeTok{ylab =} \StringTok{"Math Scores"}\NormalTok{, }\AttributeTok{trace.label =} \StringTok{"Father\textquotesingle{}s Education"}\NormalTok{)}
\end{Highlighting}
\end{Shaded}

\includegraphics{first1_files/figure-latex/unnamed-chunk-22-1.pdf}

\end{document}
