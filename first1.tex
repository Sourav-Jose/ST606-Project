% Options for packages loaded elsewhere
\PassOptionsToPackage{unicode}{hyperref}
\PassOptionsToPackage{hyphens}{url}
%
\documentclass[
]{article}
\usepackage{amsmath,amssymb}
\usepackage{iftex}
\ifPDFTeX
  \usepackage[T1]{fontenc}
  \usepackage[utf8]{inputenc}
  \usepackage{textcomp} % provide euro and other symbols
\else % if luatex or xetex
  \usepackage{unicode-math} % this also loads fontspec
  \defaultfontfeatures{Scale=MatchLowercase}
  \defaultfontfeatures[\rmfamily]{Ligatures=TeX,Scale=1}
\fi
\usepackage{lmodern}
\ifPDFTeX\else
  % xetex/luatex font selection
\fi
% Use upquote if available, for straight quotes in verbatim environments
\IfFileExists{upquote.sty}{\usepackage{upquote}}{}
\IfFileExists{microtype.sty}{% use microtype if available
  \usepackage[]{microtype}
  \UseMicrotypeSet[protrusion]{basicmath} % disable protrusion for tt fonts
}{}
\makeatletter
\@ifundefined{KOMAClassName}{% if non-KOMA class
  \IfFileExists{parskip.sty}{%
    \usepackage{parskip}
  }{% else
    \setlength{\parindent}{0pt}
    \setlength{\parskip}{6pt plus 2pt minus 1pt}}
}{% if KOMA class
  \KOMAoptions{parskip=half}}
\makeatother
\usepackage{xcolor}
\usepackage[margin=1in]{geometry}
\usepackage{color}
\usepackage{fancyvrb}
\newcommand{\VerbBar}{|}
\newcommand{\VERB}{\Verb[commandchars=\\\{\}]}
\DefineVerbatimEnvironment{Highlighting}{Verbatim}{commandchars=\\\{\}}
% Add ',fontsize=\small' for more characters per line
\usepackage{framed}
\definecolor{shadecolor}{RGB}{248,248,248}
\newenvironment{Shaded}{\begin{snugshade}}{\end{snugshade}}
\newcommand{\AlertTok}[1]{\textcolor[rgb]{0.94,0.16,0.16}{#1}}
\newcommand{\AnnotationTok}[1]{\textcolor[rgb]{0.56,0.35,0.01}{\textbf{\textit{#1}}}}
\newcommand{\AttributeTok}[1]{\textcolor[rgb]{0.13,0.29,0.53}{#1}}
\newcommand{\BaseNTok}[1]{\textcolor[rgb]{0.00,0.00,0.81}{#1}}
\newcommand{\BuiltInTok}[1]{#1}
\newcommand{\CharTok}[1]{\textcolor[rgb]{0.31,0.60,0.02}{#1}}
\newcommand{\CommentTok}[1]{\textcolor[rgb]{0.56,0.35,0.01}{\textit{#1}}}
\newcommand{\CommentVarTok}[1]{\textcolor[rgb]{0.56,0.35,0.01}{\textbf{\textit{#1}}}}
\newcommand{\ConstantTok}[1]{\textcolor[rgb]{0.56,0.35,0.01}{#1}}
\newcommand{\ControlFlowTok}[1]{\textcolor[rgb]{0.13,0.29,0.53}{\textbf{#1}}}
\newcommand{\DataTypeTok}[1]{\textcolor[rgb]{0.13,0.29,0.53}{#1}}
\newcommand{\DecValTok}[1]{\textcolor[rgb]{0.00,0.00,0.81}{#1}}
\newcommand{\DocumentationTok}[1]{\textcolor[rgb]{0.56,0.35,0.01}{\textbf{\textit{#1}}}}
\newcommand{\ErrorTok}[1]{\textcolor[rgb]{0.64,0.00,0.00}{\textbf{#1}}}
\newcommand{\ExtensionTok}[1]{#1}
\newcommand{\FloatTok}[1]{\textcolor[rgb]{0.00,0.00,0.81}{#1}}
\newcommand{\FunctionTok}[1]{\textcolor[rgb]{0.13,0.29,0.53}{\textbf{#1}}}
\newcommand{\ImportTok}[1]{#1}
\newcommand{\InformationTok}[1]{\textcolor[rgb]{0.56,0.35,0.01}{\textbf{\textit{#1}}}}
\newcommand{\KeywordTok}[1]{\textcolor[rgb]{0.13,0.29,0.53}{\textbf{#1}}}
\newcommand{\NormalTok}[1]{#1}
\newcommand{\OperatorTok}[1]{\textcolor[rgb]{0.81,0.36,0.00}{\textbf{#1}}}
\newcommand{\OtherTok}[1]{\textcolor[rgb]{0.56,0.35,0.01}{#1}}
\newcommand{\PreprocessorTok}[1]{\textcolor[rgb]{0.56,0.35,0.01}{\textit{#1}}}
\newcommand{\RegionMarkerTok}[1]{#1}
\newcommand{\SpecialCharTok}[1]{\textcolor[rgb]{0.81,0.36,0.00}{\textbf{#1}}}
\newcommand{\SpecialStringTok}[1]{\textcolor[rgb]{0.31,0.60,0.02}{#1}}
\newcommand{\StringTok}[1]{\textcolor[rgb]{0.31,0.60,0.02}{#1}}
\newcommand{\VariableTok}[1]{\textcolor[rgb]{0.00,0.00,0.00}{#1}}
\newcommand{\VerbatimStringTok}[1]{\textcolor[rgb]{0.31,0.60,0.02}{#1}}
\newcommand{\WarningTok}[1]{\textcolor[rgb]{0.56,0.35,0.01}{\textbf{\textit{#1}}}}
\usepackage{graphicx}
\makeatletter
\def\maxwidth{\ifdim\Gin@nat@width>\linewidth\linewidth\else\Gin@nat@width\fi}
\def\maxheight{\ifdim\Gin@nat@height>\textheight\textheight\else\Gin@nat@height\fi}
\makeatother
% Scale images if necessary, so that they will not overflow the page
% margins by default, and it is still possible to overwrite the defaults
% using explicit options in \includegraphics[width, height, ...]{}
\setkeys{Gin}{width=\maxwidth,height=\maxheight,keepaspectratio}
% Set default figure placement to htbp
\makeatletter
\def\fps@figure{htbp}
\makeatother
\setlength{\emergencystretch}{3em} % prevent overfull lines
\providecommand{\tightlist}{%
  \setlength{\itemsep}{0pt}\setlength{\parskip}{0pt}}
\setcounter{secnumdepth}{-\maxdimen} % remove section numbering
\usepackage{booktabs}
\usepackage{longtable}
\usepackage{array}
\usepackage{multirow}
\usepackage{wrapfig}
\usepackage{float}
\usepackage{colortbl}
\usepackage{pdflscape}
\usepackage{tabu}
\usepackage{threeparttable}
\usepackage{threeparttablex}
\usepackage[normalem]{ulem}
\usepackage{makecell}
\usepackage{xcolor}
\ifLuaTeX
  \usepackage{selnolig}  % disable illegal ligatures
\fi
\usepackage{bookmark}
\IfFileExists{xurl.sty}{\usepackage{xurl}}{} % add URL line breaks if available
\urlstyle{same}
\hypersetup{
  pdftitle={try1},
  pdfauthor={Sourav Jose},
  hidelinks,
  pdfcreator={LaTeX via pandoc}}

\title{try1}
\author{Sourav Jose}
\date{2025-03-26}

\begin{document}
\maketitle

\subsection{R Markdown}\label{r-markdown}

This is an R Markdown document. Markdown is a simple formatting syntax
for authoring HTML, PDF, and MS Word documents. For more details on
using R Markdown see \url{http://rmarkdown.rstudio.com}.

When you click the \textbf{Knit} button a document will be generated
that includes both content as well as the output of any embedded R code
chunks within the document. You can embed an R code chunk like this:

\subsection{Including Plots}\label{including-plots}

\begin{Shaded}
\begin{Highlighting}[]
\FunctionTok{library}\NormalTok{(learningtower)}
\end{Highlighting}
\end{Shaded}

\begin{verbatim}
## The learningtower package (version 1.1.0) provides data from OECD PISA database between 2000 - 2022. For package size reasons, only a small subset is provided in the package. Use the function `load_student()` to access the full data.
\end{verbatim}

\begin{Shaded}
\begin{Highlighting}[]
\FunctionTok{library}\NormalTok{(tidyverse)}
\end{Highlighting}
\end{Shaded}

\begin{verbatim}
## -- Attaching core tidyverse packages ------------------------ tidyverse 2.0.0 --
## v dplyr     1.1.4     v readr     2.1.5
## v forcats   1.0.0     v stringr   1.5.1
## v ggplot2   3.5.1     v tibble    3.2.1
## v lubridate 1.9.4     v tidyr     1.3.1
## v purrr     1.0.4
\end{verbatim}

\begin{verbatim}
## -- Conflicts ------------------------------------------ tidyverse_conflicts() --
## x dplyr::filter() masks stats::filter()
## x dplyr::lag()    masks stats::lag()
## i Use the conflicted package (<http://conflicted.r-lib.org/>) to force all conflicts to become errors
\end{verbatim}

\begin{Shaded}
\begin{Highlighting}[]
\FunctionTok{library}\NormalTok{(ggplot2)}

\FunctionTok{library}\NormalTok{(randomForest)   }\CommentTok{\# For Random Forest}
\end{Highlighting}
\end{Shaded}

\begin{verbatim}
## randomForest 4.7-1.2
## Type rfNews() to see new features/changes/bug fixes.
## 
## Attaching package: 'randomForest'
## 
## The following object is masked from 'package:dplyr':
## 
##     combine
## 
## The following object is masked from 'package:ggplot2':
## 
##     margin
\end{verbatim}

\begin{Shaded}
\begin{Highlighting}[]
\FunctionTok{library}\NormalTok{(plotly)         }\CommentTok{\# For interactive plots}
\end{Highlighting}
\end{Shaded}

\begin{verbatim}
## 
## Attaching package: 'plotly'
## 
## The following object is masked from 'package:ggplot2':
## 
##     last_plot
## 
## The following object is masked from 'package:stats':
## 
##     filter
## 
## The following object is masked from 'package:graphics':
## 
##     layout
\end{verbatim}

\begin{Shaded}
\begin{Highlighting}[]
\FunctionTok{library}\NormalTok{(caret)   }
\end{Highlighting}
\end{Shaded}

\begin{verbatim}
## Loading required package: lattice
## 
## Attaching package: 'caret'
## 
## The following object is masked from 'package:purrr':
## 
##     lift
\end{verbatim}

\begin{Shaded}
\begin{Highlighting}[]
\FunctionTok{library}\NormalTok{(factoextra) }
\end{Highlighting}
\end{Shaded}

\begin{verbatim}
## Welcome! Want to learn more? See two factoextra-related books at https://goo.gl/ve3WBa
\end{verbatim}

\begin{Shaded}
\begin{Highlighting}[]
\FunctionTok{library}\NormalTok{(gghighlight)}

\FunctionTok{library}\NormalTok{(ggrepel)}

\FunctionTok{library}\NormalTok{(tsibble)}
\end{Highlighting}
\end{Shaded}

\begin{verbatim}
## Registered S3 method overwritten by 'tsibble':
##   method               from 
##   as_tibble.grouped_df dplyr
## 
## Attaching package: 'tsibble'
## 
## The following object is masked from 'package:lubridate':
## 
##     interval
## 
## The following objects are masked from 'package:base':
## 
##     intersect, setdiff, union
\end{verbatim}

\begin{Shaded}
\begin{Highlighting}[]
\FunctionTok{library}\NormalTok{(kableExtra)}
\end{Highlighting}
\end{Shaded}

\begin{verbatim}
## 
## Attaching package: 'kableExtra'
## 
## The following object is masked from 'package:dplyr':
## 
##     group_rows
\end{verbatim}

\begin{Shaded}
\begin{Highlighting}[]
\FunctionTok{library}\NormalTok{(broom)}
\NormalTok{student }\OtherTok{\textless{}{-}} \FunctionTok{load\_student}\NormalTok{(}\StringTok{"all"}\NormalTok{)}
\end{Highlighting}
\end{Shaded}

\begin{verbatim}
## Downloading year 2000...
## 
## Downloading year 2003...
## 
## Downloading year 2006...
## 
## Downloading year 2009...
## 
## Downloading year 2012...
## 
## Downloading year 2015...
## 
## Downloading year 2018...
## 
## Downloading year 2022...
\end{verbatim}

\begin{Shaded}
\begin{Highlighting}[]
\FunctionTok{data}\NormalTok{(countrycode)}

\FunctionTok{theme\_set}\NormalTok{(}\FunctionTok{theme\_classic}\NormalTok{(}\DecValTok{18}\NormalTok{) }\SpecialCharTok{+}
            \FunctionTok{theme}\NormalTok{(}\AttributeTok{legend.position =} \StringTok{"bottom"}\NormalTok{))}
\end{Highlighting}
\end{Shaded}

\begin{Shaded}
\begin{Highlighting}[]
\NormalTok{ireland\_data }\OtherTok{\textless{}{-}}\NormalTok{ student }\SpecialCharTok{\%\textgreater{}\%} \FunctionTok{filter}\NormalTok{(country }\SpecialCharTok{==} \StringTok{"IRL"}\NormalTok{)}
\NormalTok{ireland\_data}
\end{Highlighting}
\end{Shaded}

\begin{verbatim}
## # A tibble: 36,439 x 22
##     year country school_id student_id mother_educ father_educ gender computer
##    <int> <fct>   <chr>          <int> <fct>       <fct>       <fct>  <fct>   
##  1  2000 IRL     1014               2 <NA>        <NA>        female <NA>    
##  2  2000 IRL     1014               3 <NA>        <NA>        female <NA>    
##  3  2000 IRL     1014               4 <NA>        <NA>        female <NA>    
##  4  2000 IRL     1014               5 <NA>        <NA>        female <NA>    
##  5  2000 IRL     1014               6 <NA>        <NA>        female <NA>    
##  6  2000 IRL     1014               7 <NA>        <NA>        female <NA>    
##  7  2000 IRL     1014              10 <NA>        <NA>        female <NA>    
##  8  2000 IRL     1014              12 <NA>        <NA>        female <NA>    
##  9  2000 IRL     1014              13 <NA>        <NA>        female <NA>    
## 10  2000 IRL     1014              14 <NA>        <NA>        female <NA>    
## # i 36,429 more rows
## # i 14 more variables: internet <fct>, math <dbl>, read <dbl>, science <dbl>,
## #   stu_wgt <dbl>, desk <fct>, room <fct>, dishwasher <fct>, television <fct>,
## #   computer_n <fct>, car <fct>, book <fct>, wealth <dbl>, escs <dbl>
\end{verbatim}

\begin{Shaded}
\begin{Highlighting}[]
\NormalTok{years\_in\_data }\OtherTok{\textless{}{-}} \FunctionTok{unique}\NormalTok{(ireland\_data}\SpecialCharTok{$}\NormalTok{year)}
\FunctionTok{print}\NormalTok{(years\_in\_data)}
\end{Highlighting}
\end{Shaded}

\begin{verbatim}
## [1] 2000 2003 2006 2009 2012 2015 2018 2022
\end{verbatim}

\begin{Shaded}
\begin{Highlighting}[]
\NormalTok{ireland\_data }\OtherTok{\textless{}{-}}\NormalTok{ student }\SpecialCharTok{\%\textgreater{}\%}
  \FunctionTok{filter}\NormalTok{(country }\SpecialCharTok{==} \StringTok{"IRL"} \SpecialCharTok{\&}\NormalTok{ year }\SpecialCharTok{!=} \DecValTok{2000}\NormalTok{)}


\NormalTok{(ireland\_data)}
\end{Highlighting}
\end{Shaded}

\begin{verbatim}
## # A tibble: 34,305 x 22
##     year country school_id student_id mother_educ    father_educ gender computer
##    <int> <fct>   <chr>          <int> <fct>          <fct>       <fct>  <fct>   
##  1  2003 IRL     00001              1 ISCED 3A       ISCED 3A    female yes     
##  2  2003 IRL     00001              2 ISCED 3A       ISCED 3A    male   yes     
##  3  2003 IRL     00001              3 ISCED 2        ISCED 2     male   no      
##  4  2003 IRL     00001              4 ISCED 3A       ISCED 1     female yes     
##  5  2003 IRL     00001              5 ISCED 3A       ISCED 3A    male   yes     
##  6  2003 IRL     00001              6 less than ISC~ less than ~ male   yes     
##  7  2003 IRL     00001              7 ISCED 3A       ISCED 2     male   yes     
##  8  2003 IRL     00001              8 ISCED 3A       ISCED 3A    female yes     
##  9  2003 IRL     00001              9 ISCED 3A       ISCED 3A    male   yes     
## 10  2003 IRL     00001             10 ISCED 1        ISCED 3A    female yes     
## # i 34,295 more rows
## # i 14 more variables: internet <fct>, math <dbl>, read <dbl>, science <dbl>,
## #   stu_wgt <dbl>, desk <fct>, room <fct>, dishwasher <fct>, television <fct>,
## #   computer_n <fct>, car <fct>, book <fct>, wealth <dbl>, escs <dbl>
\end{verbatim}

\begin{Shaded}
\begin{Highlighting}[]
\FunctionTok{levels}\NormalTok{(ireland\_data}\SpecialCharTok{$}\NormalTok{father\_educ) }\OtherTok{\textless{}{-}} \FunctionTok{c}\NormalTok{(}\StringTok{"less than ISCED1"}\NormalTok{, }\StringTok{"ISCED 1"}\NormalTok{,}\StringTok{"ISCED 2"}\NormalTok{,}\StringTok{"ISCED 3A"}\NormalTok{,}\StringTok{"ISCED 3B, C"}\NormalTok{ )}
\end{Highlighting}
\end{Shaded}

\begin{Shaded}
\begin{Highlighting}[]
\NormalTok{ireland\_data }\SpecialCharTok{\%\textgreater{}\%}
  \FunctionTok{group\_by}\NormalTok{(year, gender) }\SpecialCharTok{\%\textgreater{}\%}
  \FunctionTok{summarise}\NormalTok{(}\AttributeTok{mean\_math =} \FunctionTok{mean}\NormalTok{(math, }\AttributeTok{na.rm =} \ConstantTok{TRUE}\NormalTok{)) }\SpecialCharTok{\%\textgreater{}\%}
  \FunctionTok{ggplot}\NormalTok{(}\FunctionTok{aes}\NormalTok{(}\AttributeTok{x =}\NormalTok{ year, }\AttributeTok{y =}\NormalTok{ mean\_math, }\AttributeTok{color =}\NormalTok{ gender)) }\SpecialCharTok{+}
  \FunctionTok{geom\_line}\NormalTok{() }\SpecialCharTok{+} \FunctionTok{geom\_point}\NormalTok{() }\SpecialCharTok{+}
  \FunctionTok{labs}\NormalTok{(}\AttributeTok{title =} \StringTok{"Math Score Trends by Gender"}\NormalTok{, }\AttributeTok{y =} \StringTok{"Mean Math Score"}\NormalTok{, }\AttributeTok{x =} \StringTok{"Year"}\NormalTok{)}
\end{Highlighting}
\end{Shaded}

\begin{verbatim}
## `summarise()` has grouped output by 'year'. You can override using the
## `.groups` argument.
\end{verbatim}

\includegraphics{first1_files/figure-latex/unnamed-chunk-6-1.pdf}

\begin{Shaded}
\begin{Highlighting}[]
\CommentTok{\# Boxplot of math scores by computer access}
\FunctionTok{ggplot}\NormalTok{(ireland\_data, }\FunctionTok{aes}\NormalTok{(}\AttributeTok{x =} \FunctionTok{factor}\NormalTok{(computer\_n), }\AttributeTok{y =}\NormalTok{ math, }\AttributeTok{fill =} \FunctionTok{factor}\NormalTok{(computer\_n))) }\SpecialCharTok{+}
  \FunctionTok{geom\_boxplot}\NormalTok{() }\SpecialCharTok{+}
  \FunctionTok{scale\_fill\_manual}\NormalTok{(}\AttributeTok{values =} \FunctionTok{c}\NormalTok{(}\StringTok{\textquotesingle{}blue\textquotesingle{}}\NormalTok{, }\StringTok{\textquotesingle{}red\textquotesingle{}}\NormalTok{, }\StringTok{\textquotesingle{}green\textquotesingle{}}\NormalTok{, }\StringTok{\textquotesingle{}yellow\textquotesingle{}}\NormalTok{))}\SpecialCharTok{+}
  \FunctionTok{theme\_minimal}\NormalTok{() }\SpecialCharTok{+}
  \FunctionTok{labs}\NormalTok{(}\AttributeTok{title =} \StringTok{"Math Scores Distribution by Computer Access"}\NormalTok{,}
       \AttributeTok{x =} \StringTok{"Computer Access "}\NormalTok{,}
       \AttributeTok{y =} \StringTok{"Math Scores"}\NormalTok{) }\SpecialCharTok{+}
  \FunctionTok{scale\_x\_discrete}\NormalTok{()}
\end{Highlighting}
\end{Shaded}

\includegraphics{first1_files/figure-latex/unnamed-chunk-7-1.pdf}

\begin{Shaded}
\begin{Highlighting}[]
\NormalTok{ireland\_data }\SpecialCharTok{\%\textgreater{}\%}
  \FunctionTok{group\_by}\NormalTok{(year, internet) }\SpecialCharTok{\%\textgreater{}\%}
  \FunctionTok{summarise}\NormalTok{(}\AttributeTok{mean\_math =} \FunctionTok{mean}\NormalTok{(math, }\AttributeTok{na.rm =} \ConstantTok{TRUE}\NormalTok{)) }\SpecialCharTok{\%\textgreater{}\%}
  \FunctionTok{ggplot}\NormalTok{(}\FunctionTok{aes}\NormalTok{(}\AttributeTok{x =}\NormalTok{ year, }\AttributeTok{y =}\NormalTok{ mean\_math, }\AttributeTok{color =}\NormalTok{ internet)) }\SpecialCharTok{+}
  \FunctionTok{geom\_line}\NormalTok{() }\SpecialCharTok{+} \FunctionTok{geom\_point}\NormalTok{() }\SpecialCharTok{+}
  \FunctionTok{labs}\NormalTok{(}\AttributeTok{title =} \StringTok{"Math Scores by Internet Access Over Years"}\NormalTok{, }\AttributeTok{y =} \StringTok{"Mean Math Score"}\NormalTok{, }\AttributeTok{x =} \StringTok{"Year"}\NormalTok{)}
\end{Highlighting}
\end{Shaded}

\begin{verbatim}
## `summarise()` has grouped output by 'year'. You can override using the
## `.groups` argument.
\end{verbatim}

\includegraphics{first1_files/figure-latex/unnamed-chunk-8-1.pdf}

\begin{Shaded}
\begin{Highlighting}[]
\CommentTok{\# Boxplot for father\textquotesingle{}s education vs. math scores}
\FunctionTok{boxplot}\NormalTok{(math }\SpecialCharTok{\textasciitilde{}}\NormalTok{ father\_educ, }\AttributeTok{data =}\NormalTok{ ireland\_data, }
        \AttributeTok{main =} \StringTok{"Math Scores by Father\textquotesingle{}s Education Level"}\NormalTok{,}
        \AttributeTok{xlab =} \StringTok{"Father\textquotesingle{}s Education Level"}\NormalTok{, }\AttributeTok{ylab =} \StringTok{"Math Score"}\NormalTok{,}
        \AttributeTok{col =} \StringTok{"lightcoral"}\NormalTok{, }\AttributeTok{border =} \StringTok{"darkred"}\NormalTok{)}
\end{Highlighting}
\end{Shaded}

\includegraphics{first1_files/figure-latex/unnamed-chunk-9-1.pdf}

\begin{Shaded}
\begin{Highlighting}[]
\NormalTok{data\_filtered }\OtherTok{\textless{}{-}}\NormalTok{ ireland\_data }\SpecialCharTok{\%\textgreater{}\%} \FunctionTok{filter}\NormalTok{(}\SpecialCharTok{!}\FunctionTok{is.na}\NormalTok{(book))}

\NormalTok{data\_filtered}\SpecialCharTok{$}\NormalTok{book }\OtherTok{\textless{}{-}} \FunctionTok{factor}\NormalTok{(data\_filtered}\SpecialCharTok{$}\NormalTok{book, }\AttributeTok{levels =} \FunctionTok{c}\NormalTok{(}
  \StringTok{"0{-}10"}\NormalTok{, }\StringTok{"11{-}25"}\NormalTok{, }\StringTok{"26{-}100"}\NormalTok{, }\StringTok{"101{-}200"}\NormalTok{, }\StringTok{"201{-}500"}\NormalTok{, }\StringTok{"more than 500"}
\NormalTok{))}
\FunctionTok{ggplot}\NormalTok{(data\_filtered, }\FunctionTok{aes}\NormalTok{(}\AttributeTok{x =}\NormalTok{ book, }\AttributeTok{y =}\NormalTok{ math, }\AttributeTok{fill =}\NormalTok{ book)) }\SpecialCharTok{+}
  \FunctionTok{geom\_boxplot}\NormalTok{() }\SpecialCharTok{+}
  \FunctionTok{labs}\NormalTok{(}\AttributeTok{title =} \StringTok{"Distribution of Math Scores by Book Availability"}\NormalTok{,}
       \AttributeTok{x =} \StringTok{"Book Availability"}\NormalTok{,}
       \AttributeTok{y =} \StringTok{"Math Score"}\NormalTok{) }\SpecialCharTok{+}
  \FunctionTok{scale\_fill\_brewer}\NormalTok{(}\AttributeTok{palette =} \StringTok{"Set3"}\NormalTok{) }\SpecialCharTok{+}  
  \FunctionTok{theme\_minimal}\NormalTok{()}
\end{Highlighting}
\end{Shaded}

\includegraphics{first1_files/figure-latex/unnamed-chunk-10-1.pdf}

\begin{Shaded}
\begin{Highlighting}[]
\FunctionTok{library}\NormalTok{(ggplot2)}

\FunctionTok{ggplot}\NormalTok{(ireland\_data, }\FunctionTok{aes}\NormalTok{(}\AttributeTok{x =}\NormalTok{ escs, }\AttributeTok{y =}\NormalTok{ math)) }\SpecialCharTok{+}
  \FunctionTok{geom\_hex}\NormalTok{(}\AttributeTok{bins =} \DecValTok{50}\NormalTok{) }\SpecialCharTok{+}
  \FunctionTok{scale\_fill\_viridis\_c}\NormalTok{(}\AttributeTok{option =} \StringTok{"plasma"}\NormalTok{) }\SpecialCharTok{+}  \CommentTok{\# colorful scale for counts}
  \FunctionTok{geom\_smooth}\NormalTok{(}\AttributeTok{method =} \StringTok{"lm"}\NormalTok{, }\AttributeTok{color =} \StringTok{"red"}\NormalTok{) }\SpecialCharTok{+}
  \FunctionTok{labs}\NormalTok{(}\AttributeTok{title =} \StringTok{"Hexbin Plot of Math Scores vs. ESCS"}\NormalTok{,}
       \AttributeTok{x =} \StringTok{"ESCS"}\NormalTok{, }\AttributeTok{y =} \StringTok{"Math Score"}\NormalTok{, }\AttributeTok{fill =} \StringTok{"Count"}\NormalTok{) }\SpecialCharTok{+}
  \FunctionTok{theme\_minimal}\NormalTok{()}
\end{Highlighting}
\end{Shaded}

\begin{verbatim}
## Warning: Removed 454 rows containing non-finite outside the scale range
## (`stat_binhex()`).
\end{verbatim}

\begin{verbatim}
## `geom_smooth()` using formula = 'y ~ x'
\end{verbatim}

\begin{verbatim}
## Warning: Removed 454 rows containing non-finite outside the scale range
## (`stat_smooth()`).
\end{verbatim}

\includegraphics{first1_files/figure-latex/unnamed-chunk-11-1.pdf}

\begin{Shaded}
\begin{Highlighting}[]
\CommentTok{\# Boxplot of Math Scores by Room Availability}
\FunctionTok{ggplot}\NormalTok{(ireland\_data, }\FunctionTok{aes}\NormalTok{(}\AttributeTok{x =}\NormalTok{ room, }\AttributeTok{y =}\NormalTok{ math, }\AttributeTok{fill =}\NormalTok{ room)) }\SpecialCharTok{+}
  \FunctionTok{geom\_boxplot}\NormalTok{() }\SpecialCharTok{+}
  \FunctionTok{labs}\NormalTok{(}\AttributeTok{title =} \StringTok{"Math Scores by Room Availability"}\NormalTok{, }\AttributeTok{x =} \StringTok{"Room Availability"}\NormalTok{, }\AttributeTok{y =} \StringTok{"Math Score"}\NormalTok{) }\SpecialCharTok{+}
  \FunctionTok{scale\_fill\_manual}\NormalTok{(}\AttributeTok{values =} \FunctionTok{c}\NormalTok{(}\StringTok{"lightblue"}\NormalTok{, }\StringTok{"lightgreen"}\NormalTok{)) }\SpecialCharTok{+}
  \FunctionTok{theme\_minimal}\NormalTok{()}
\end{Highlighting}
\end{Shaded}

\includegraphics{first1_files/figure-latex/unnamed-chunk-12-1.pdf}

\end{document}
